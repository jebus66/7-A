\documentclass[a4paper,10pt]{article}
%\documentclass[a4paper,10pt]{scrartcl}
\usepackage[spanish]{babel}
\usepackage[utf8]{inputenc}
\usepackage{amssymb, amsmath, amsbsy}
\usepackage{cancel} % para tachar
\usepackage{mathdots} % para el comando \iddots
\usepackage{mathrsfs} % para formato de letra
\usepackage{stackrel} % para el comando \stackbin

\title{Investigacion de rotacion y cuaternios}
\author{Gutiérrez Muñoz José de Jesús \\ 7 - A \\ Ing. Mecatrónica}
\date{16 - Septiembre - 2019}

\begin{document}
\maketitle
\large
Rotación.


De los recursos de geometría analítica recordamos que al rotar un vector (x,y) $\in \mathbb{R}^2$ del origen por un ángulo $\theta$ obtenemos otro vector $({x'},{y'}) \in \mathbb{R}^2$ cuyas coordenadas son:

$x{'}= cos(0)x−sen(0)y, y{'}= sen(0)x+ cos(0)y$

Lo cual expresamos por:
\begin{equation}
\left
[\begin{array}{lcr}
x{'}\\ y{'} \\
\end{array}\right]
R = \theta
\left
[\begin{array}{lcr}
x\\ y \\
\end{array}\right]
\end{equation}

Donde:

\begin{equation}
R ( \theta ) =
\left
[\begin{array}{lcr}
cos \ 0 & -sen \ 0\\ sen \ 0 & cos \ 0\\
\end{array}\right]
\end{equation}

Observemos que:

\begin{equation}
R( \theta ) \ . \ R^T( \theta ) =
\left
[\begin{array}{lcr}
1 \ & \ 0 \\ 0 \ & \ 1 \\
\end{array}\right]
\end{equation}

por lo que R $\in$ SO(2).


La froma de mitad de ángulo. Podemos obterner la matriz de rotación para un vector en el plano como sigue. \\ Consideremos:

\begin{equation}
A = 
\left
[\begin{array}{lcr}
a^2-b^2
\end{array}\right]
,B =
\left
[\begin{array}{lcr}
2ab
\end{array}\right]
\end{equation}

Si  $A =cos \frac{0}{2}$ y  $B =sen \frac{0}{2}$ tenemos que

\begin{equation}
A =
\left
[\begin{array}{lcr}
cos^2(\frac{0}{2})-sen^2(\frac{0}{2})
\end{array}\right]
\end{equation}

\begin{equation} 
=
\left
[\begin{array}{lcr}
1 - 2 sen^2 (\frac{0}{2})
\end{array}\right]
\end{equation}
\begin{equation}
=
\left
[\begin{array}{lcr}
cos \ 0
\end{array}
\right]
\end{equation}
\begin{equation}
B =
\left
[\begin{array}{lcr}
2cos  (\frac{0}{2}) sen(\frac{0}{2})
\end{array}\right]
\end{equation}
\begin{equation}
=
\left
[\begin{array}{lcr}
sen \ 0
\end{array}\right]
\end{equation}

de modo que la matriz de rotación no cambia

\begin{equation}
R_x(\alpha) =
\left
[\begin{array}{lcr}
1 \ \ \ \ \ \ 0 \ \ \ \ \ \ \ \ \ 0 \\ 0 \ \ cos \ \alpha \ \ -sen \ \alpha \\ 0 \ \ \ \ sen \ \alpha \ \ \ \ cos \ \alpha
\end{array}\right],
\end{equation}

\begin{equation}
R_y(\phi) =
\left
[\begin{array}{lcr}
cos \ \phi \ \ \ \ \ \ \ \ 0 \ \ \ \ \ \ sen \ \phi \\ \ \ \ \ 0 \ \ \ \ \ \ \ \ \ 1 \ \ \ \ \ \ \ \ 0 \\ -sen \ \phi \ \ \ \ \ 0 \ \ \ \ cos \ \phi
\end{array}\right],
\end{equation}

\begin{equation}
R_z(\theta) =
\left
[\begin{array}{lcr}
cos \ \theta \ -sen \ \theta \ \ \ \ 0 \\ sen \ \theta \ \ \ cos \ \theta \ \ \ \ \ \ 0 \\ \ \ \ \ 0 \ \ \ \ \ \ \ 0 \ \ \ \ \ \ \ \ \ 1
\end{array}\right],
\end{equation}

En este caso se tiene:

$R_x(\alpha)$ rota el plano yz alrededor del origen por un ángulo $\alpha$

$R_y(\phi)$ rota el plano xz alrededor del origen por un ángulo $\phi$

$R_z(\theta)$ rota el plano xy alrededor del origen por un ángulo $\theta$


\Large
Cuaternios.


Para enetebnder la relación entre rotaciones $\mathbb{R}^3$ y los cuaternios, es conveniente utilizar la notación $q = [\lambda,a]$ para el cuaternios $q = \lambda + xi + ji + zk$, donde $a = xi + yj + zk$. La separación del cuaternio q en dos partes nos permite distinguir su parte real $\lambda$ y su ``Parte imaginaria'' a; además, identificamos el cuaternio puro $a = xi + yj + zk$ con el vector $a = (x,y,z)\in \mathbb{R}^3$. De manera recíproca, dado un vector $b = (u,v,w)\in \mathbb{R}^3$, le hacemos corresponder el cuaternio $q_b = [0,b]$, con $b = ui + vj + wk \in \mathbb{H}_p$.


De esta manera, dados los cuaternio $q = [\lambda,a]$ y $r = [\mu,b]$ definimos dpos operaciones entre las partes imaginarias de q y r: El producto punto y el producto cruz de los vectores $a,b \in \mathbb{R}^3$.


Notemos que con esta definición, {a,b} $\in \mathbb{R}$ y a x b es otro cuaternio con parte real igual a cero y su ``parte imaginaria'' está dada por los componentes del vector a x b. Usando esta notación se tiene que el producto de los cuaternio $q = [\lambda,a]$ y $r = [\mu,b]$ se expresa como:

\begin{equation}
 q . r = [\lambda,a] . [\mu,b] = [\lambda \mu - {a,b}, \lambda b + \mu a + a x b,
\end{equation}


Y esta fórmula nos será muy útil para describir la relación entre los cuaternios unitarios y las rotaciones en $\mathbb{R}^3$.

\end{document}
